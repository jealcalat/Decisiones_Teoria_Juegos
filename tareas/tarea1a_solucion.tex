\documentclass[12pt]{scrartcl} % se puede cambiar por article
\RequirePackage{amsmath,amsfonts,amssymb,amsthm}
\usepackage{eulervm} % tipografía con soporte para matemáticas
\usepackage{caption}

\usepackage{exercise}
\usepackage{graphicx}
\DeclareMathOperator*{\argmax}{argmax}

\usepackage[utf8]{inputenc}
\usepackage[spanish,mexico]{babel}
% sgamex.sty from Rubinstein
\usepackage{sgamex}

\usepackage{setspace}
\onehalfspacing % doble espacio: \doublespacing

\title{Solución de la tarea 1a}
\author{Emmanuel Alcalá}
\date{\today}

\begin{document}
\maketitle

% Respuestas enumeradas con paquete Exercise

\begin{Exercise}[name={Respuesta}]

  \[\mathbf{E}[X] = \sum_{i=1}^n p(x_i)x_i = \mathbf{p\cdot x} =%
  \begin{bmatrix} 
    0.01 & 0.1 & 0.1 & 0.09 & 0.2 & 0.5
  \end{bmatrix}  
    \begin{bmatrix}
    0 \\ 
    1 \\
    2 \\
    4 \\
    10 \\
    5
    \end{bmatrix} = 5.16 \]

\end{Exercise}

\begin{Exercise}[name={Respuesta}]



Comparar la utilidad de \textbf{No defraudar} con la \textit{utilidad esperada} de \textbf{Defraudar}:

\begin{enumerate}
  \setlength{\itemsep}{0pt}
  \setlength{\parskip}{0pt}
  \setlength{\parsep}{0pt}
  \item R. 
  \begin{center}
    \includegraphics[scale=0.65]{p2tarea1a.png}
  \end{center}
  \item R. Averso. Su función es cóncava. 
  \item R. 
  \begin{align*}
    U(ND) &= 9.486\\
    UE(D) &= 0.95 \times 9.866 + 0.05 \times 3.796 = 9.562
  \end{align*}
  
  La utilidad por \textbf{No defraudar} es \textit{menor} que la utilidad por \textbf{Defraudar}.
  \item R. 
  \begin{align*}
    U(ND) &\geq UE(D) \\ 
    9.486 &\geq (1-p) \times 9.866 + p \times 3.796 \\ 
    9.456-9.866 &\geq 3.79p - 9.866p \\
    -0.41 &\geq -6.076p\\
    \text{cambiar } &\text{de signo} \\
    p &\geq \frac{0.41}{6.076} \\
    p &\geq 0.067
  \end{align*}
 
Si la probabilidad es mayor a 0.067 (o dicho de otra manera, si la proporción de auditados por SAT es mayor al 6.7 \%), sería racional para el individuo cambiar de decisión y no defraudar.

\end{enumerate} 

\end{Exercise}

\begin{Exercise}[name={Respuesta}]

 \begin{enumerate}
   \setlength{\itemsep}{0pt}
   \setlength{\parskip}{0pt}
   \setlength{\parsep}{0pt}

   \item R. El juego en su forma normal es:
   \begin{align*}
     N &= \{J1, J2\}\\
     S &= q_i \in [0, \infty) \\
     u_i(q_i, q_j) &= (100  - q_i - q_j)q_i - q_i^2\\
   \end{align*}
   \item R. Primero definimos el problema como un problema de optimización. \[q_i^* = \argmax_{q_i > 0} u_i(q_i, q_j^*)\]
   Dado que la función de utilidad involucra un término cuadrado en negativo, sabemos que es cóncava, por lo que el C.P.O. es suficiente:
   \[\frac{\partial u_i(q_i, q_j)}{\partial q_i} \Bigr|_{q_j = q_j^*} = 0\]
   \begin{align*}
    \frac{\partial u_i(q_i, q_j)}{\partial q_i} \Bigr|_{q_j = q_j^*} &= 0\\
    \frac{\partial}{\partial q_i} (100q_i-2q_i^2 -q_iq_j) &= 0 \\
    100 -4q_i -q_j &= 0 \\
    % q_i^* &= \frac{100-q_j}{4}\\
   \end{align*}
   Lo que resulta en un sistema de ecuaciones de dos incógnitas:
   \begin{align*}
    100 -4q_1 -q_2 &= 0\\
    100 -q_1 - 4q_2 &= 0\\
   \end{align*}
   Resolviendo
   \begin{align*}
    \begin{cases}
    (100 + 4q_1 -q_2 &= 0)\times (-4)\\
    100 + q_1 - 4q_2 &= 0
  \end{cases} 
    \iff  \begin{cases}
      -400 + 16q_1 + 4q_2 &= 0\\
       100 -q_1 - 4q_2    &= 0\\%[-0.2ex]
       \multispan2{\hrulefill}\\
       -300 + 15q_1 & = 0
  \end{cases}\\
  q_i^* = \frac{300}{15} = 20
   \end{align*}
   El EN es $ (q_1^* = 20, q_2^* = 20) $, con ganancias en equilibrio de $ (800, 800) $.
 \end{enumerate}

\end{Exercise}

\end{document}

