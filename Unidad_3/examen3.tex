\PassOptionsToPackage{table}{xcolor}
\documentclass[12pt]{scrartcl} % se puede cambiar por article
\RequirePackage{amsmath,amsfonts,amssymb,amsthm}
\usepackage{eulervm} % tipografía con soporte para matemáticas
\usepackage{caption}
\usepackage{exercise}
\usepackage{tikzsymbols}

\usepackage[utf8]{inputenc}
\usepackage[spanish,mexico]{babel}
% sgamex.sty from Rubinstein
\usepackage{sgamex}
\usepackage{dtj_iteso} % creación mía

\usepackage{setspace}
\onehalfspacing % doble espacio: \doublespacing

\title{Examen - Unidad 3 \\ \normalsize Decisiones y Teoría de Juegos}
\author{Emmanuel Alcalá\\ \url{jaime.alcala@iteso.mx}}
\date{\today}


\begin{document}
\maketitle
\hrule

\begin{summarybox}{Instrucciones}

    % \begin{description}
    %     \item[1] - Entregar el examen en la fecha acordada en CANVAS, \textbf{\color{blue} en formato pdf}. %Puedes contestar en papel, tomar fotos y colocarlas en un archivo Word y convertirlo a pdf, luego subirlo en la entrada de CANVAS correspondiente al examen, o como te parezca apropiado, pero en \textbf{pdf}.
    %     \item[2] - Escribe \textbf{claramente} los cálculos que desarrollaste para resolver los problemas, y \textbf{encierra en un recuadro} la respuesta correcta. 
    %     \item[3] - {\color{blue} IMPORTANTE}: 1) problemas que solo contengan respuestas sin desarrollo los consideraré {\color{red} erróneos}; 2) copiar y plagiar respuestas no será tolerado, y cada respuesta identificada como copia de la respuesta de otro estudiante será {\color{red} anulada} para ambos estudiantes.
    %     \item[4] - No habrá plazos extras. Si existe una situación \textit{extraordinaria}, házmelo saber con tiempo y evidencia.
    %     \item[5] - Las únicas dudas que contestaré serán relativas a la redacción (errores o alguna confusión). %No responderé si un resultado es correcto, ni mucho menos cómo resolver algo. Los ejercicios han sido seleccionados estrictamente dentro del material revisado en clases y en las notas de clase. 
    % \end{description}
    
    \begin{description} % instrucciones para examen durante la clase
      \item[1] - El examen tendrá lugar entre la semana del 2 de mayo al 6 de mayo, de forma presencial.
      \item[2] - Formar equipos de cuatro personas. 
      \item[3] - Entregar un solo examen físico en hojas blancas el miércoles 04 de mayo con los nombres de todos los integrantes y las soluciones en limpio. 
      \item[4] - Reglas:
     \begin{itemize}
       \setlength{\itemsep}{0pt}
       \setlength{\parskip}{0pt}
       \setlength{\parsep}{0pt}
       \item Cada equipo podrá hacerle al profesor dos preguntas. Si me niego a contestarla (e.g., si me preguntan algo que no pueda contestar sin resolver el problema) pueden volver a hacer la pregunta, pero solo una vez. Piensen bien qué preguntar.
       \item Pueden consultar apuntes y libros. Prepárense de forma previa al examen para saber qué podrían necesitar o facilitar el examen. Esto pueden saberlo consultando la guía. 
       \item La fecha de entrega del examen es inaplazable. Recomiendo que, cada que estén seguros de una respuesta, vayan pasándola en limpio.
       \item La hoja de soluciones debe ser legible y ordenada.
       \item La calificación de cada ejercicio se divide equitativamente en cada inciso.
     \end{itemize}   
      
    \end{description}

\end{summarybox}

\begin{center}
  \Coffeecup[1.5]
\end{center}

% Respuestas enumeradas con paquete Exercise
\begin{Exercise}[title={Subasta de sobre cerrado con $n$ jugadores},name={Pregunta}]
\textit{7.5 pt}

  Considera la subasta de sobre cerrado al primer precio considerada en clase. Las valoraciones $x_1, x_2,...,x_n$ de los jugadores son desconocidas pero \textit{independientes} y uniformemente distribuidas entre 0 y 100. Asume que los jugadores usan una función de puja $b_i(x_i)=ax_i$.
  
  \begin{enumerate}
      \item Demuestra que $a=\frac{n-1}{n}$.
      \item Compara las pujas $n=2$ con $n=3$. ¿Qué le conviene más al vendedor, una $n$ grande o una pequeña?
      \item Demuestra que cuando $n \rightarrow \infty$, los jugadores van a pujar su valuación, es decir $b_i = x_i$.
  \end{enumerate}
  
  \textbf{Pista}: para que el jugador $ i $ gane, su puja debe ser mayor que la de cada uno de los jugadores, es decir, $ b_i > b_1, b_i > b_2, ..., b_i > b_{i-1}, b_i > b_{i+1}, b_i > b_{n-1} $, y cada uno de estos eventos es independiente. %Lo primero, y más difícil, es resolver la probabilidad de ganar. Para el cálculo de la probabilidad, toma en cuenta que cada puja de los otros $ n-1 $ jugadores es independiente, por lo que, asumiendo, $ i \neq j $, el evento en el que el jugador $ i $ puja más alto que el jugador $j$ es independiente del evento en el que el jugadores $ i $ puja más alto que el jugador $ j+1 $, y así por el estilo.

  % \textbf{Hints:} 1) Asume que los jugadores usan una función de puja $b_i=ax_i$; 2) para $n-1$ jugadores (es decir, $n$ menos el jugador $i$, dado que obtendremos el resultado desde su perspectiva), la probabilidad de que las $n-1$ valuaciones de los $n-1$ jugadores sean menor que la puja del jugador $i$, o sea $y$, el producto de la probabilidad de un solo jugador multiplicada $n-1$ veces por sí misma. Si la probabilidad de $j$ es $p(x_j < y/a)=\frac{y}{100a}$, para $n-1$ jugadores es $p(x_1 < y/a)\cdot p(x_2 < y/a) \dots p(x_{n-1} < y/a)$; 3) recuerda que, para dos jugadores, la utilidad del jugador 1 por su puja $y$ es $UE_1(y, b_2)=ganancia \times p(ganar) + 0\times p(perder)$.  
  
  
  \end{Exercise}

  \begin{Exercise}[title={Duopolio de Cournot con información asimétrica}, name={Pregunta}]
    \textit{7.5 pt}

    Considera un duopolio de Cournot con demanda inversa $P(Q) = a - Q$, en donde $Q = q_1 + q_2$ es la cantidad agregada en el mercado. Ambas empresas tienen un costo $c_i(q_i) = cq_i$, pero la demanda $a$ es desconocida: es alta ($a = a_A)$ con probabilidad $\theta$ y baja ($a=a_B$) con probabilidad $1-\theta$. Es un juego de información asimétrica. La empresa 1 sabe si la demanda es alta o baja, pero la empresa 2 no lo sabe. Las dos empresas deben escoger simultáneamente las cantidades. Contesta:
    
    \begin{enumerate}
        \item ¿Cuál es el Equilibrio de Nash Bayesiano en este juego? Es decir, ¿qué cantidades $q_i^{k*}$ deben escoger las empresas, en función de la demanda $a_k$?
        \item ¿Cómo varían las cantidades en equilibrio para cada empresa con respecto a $ \theta \text{ y } a_A$?
        \item Para la empresas 2, ¿cuál sería la cantidad en equilibrio si la probabilidad de demanda alta fuese de 0?
    \end{enumerate}
    
    % \textbf{Hint}: La empresa 2 no sabe los tipos de la empresa 1, pero conoce su propio tipo. Asume que la empresa 2 solo tiene un tipo, por lo que su utilidad se computa distinto que la de la empresa 1. El espacio de estrategias se construye a partir de una regla de decisión. Por ejemplo, para empresa 1, $s^*_1(a_k)=q_1^*(a_k) = ...$ dependiendo de qué valor tome $a_k$, que puede ser $a_A, a_B$. 
    \definecolor{cadet}{rgb}{0.33, 0.41, 0.47}
    \textbf{Pista:} \textbf{\color{cadet} nota que la demanda es la misma para ambas empresas} por lo que, para la empresa 2, el problema de optimización para $q_2^*$ es 
    
    $$
    \argmax_{q_2 \geq 0} ~ \{ \theta({\color{red}a_A} - q_1^{A} - q_2 - c)q_2 + (1 - \theta)({\color{red}a_B}- q_1^{B}-q_2-c)q_2 \}
    $$
    
    \end{Exercise}

  \begin{center}
    ¡Suerte! 
    
    \LARGE\Cat[1.2] 
  \end{center}


\end{document}