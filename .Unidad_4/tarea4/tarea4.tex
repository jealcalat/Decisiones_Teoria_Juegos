\PassOptionsToPackage{table}{xcolor}
\documentclass[12pt]{scrartcl} % se puede cambiar por article
\RequirePackage{amsmath,amsfonts,amssymb,amsthm}
% \usepackage{eulervm} % tipografía con soporte para matemáticas
\usepackage{caption}
\usepackage{exercise}
\usepackage{tikzsymbols}

\usepackage[utf8]{inputenc}
\usepackage[spanish,mexico]{babel}
% sgamex.sty from Rubinstein
\usepackage{sgamex}
\usepackage{dtj_iteso} % creación mía
\usepackage{kpfonts}

\usepackage{setspace}
\onehalfspacing % doble espacio: \doublespacing

\title{Tarea - Unidad 4 \\ \normalsize Decisiones y Teoría de Juegos}
\author{Emmanuel Alcalá\\ \url{jaime.alcala@iteso.mx}}
\date{\today}


\begin{document}
\maketitle
\begin{summarybox}{Instrucciones}

  \begin{description}
      \item[1] - Puedes contestar en papel, tomar fotos y colocarlas en un archivo Word y convertirlo a pdf, luego subirlo en la entrada de CANVAS correspondiente al examen. 
      \item[2] - Coloca claramente los cálculos que desarrollaste para resolver los problemas, y \textbf{encierra en un recuadro} la respuesta correcta. 
      \item[3] - Si existe una situación \textit{extraordinaria} que te impida \textit{terminar} y subir la tarea a tiempo, házmelo saber en ese momento. Sin embargo, si es el último día y no haz hecho nada, por día que pase sin que hayas subido la tarea, perderás 10\% de puntos.
  \end{description}
  
\end{summarybox}

\begin{center}
  \Coffeecup[1.5]
\end{center}

\begin{Exercise}[name=Pregunta]
\textit{3 pt}

Abajo se encuentra un juego de señalización representado de forma genérica.

\begin{center}
	\begin{tikzpicture}[scale=1.4,font=\footnotesize]
		\tikzset{%
			% Two node styles for game trees: solid and hollow
			solid node/.style={circle,draw,inner sep=1.5,fill=black},
			solid nature/.style={circle,draw=black,inner sep=2.5},
			hollow node/.style={circle,draw,inner sep=1.5}
		}
		% Specify spacing for each level of the tree
		\tikzstyle{level 1}=[level distance=12mm,sibling distance=25mm]
		\tikzstyle{level 2}=[level distance=15mm,sibling distance=15mm]
		\tikzstyle{level 3}=[level distance=17mm,sibling distance=10mm]
		% The Tree
		\node(0)[solid nature,label=right:{N}]{}
		child[grow=up]{
		% node[solid node,label=above:{
		% \begin{tabular}{c}
		% 	Sender \\ $t=t_1$
		% \end{tabular}
		% }] {}
		node[solid node,label=above:{Emisor}] {}
		node[solid node,label=below left:{\rotatebox{90}{$t_2$}}] {}
		child[grow=left]{
		node(1)[solid node,label=below:{$[p]$}]{}
		child{node[hollow node,label=left:{}]{} edge from parent node [above]{$a_1$}}
		child{node[hollow node,label=left:{}]{} edge from parent node [below]{$a_2$}}
		edge from parent node [above]{$m_1$}
		}
		child[grow=right]{node(3)[solid node,label=below:{$[q]$}]{}
		child{node[hollow node,label=right:{}]{} edge from parent node [below]{$a_2$}}
		child{node[hollow node,label=right:{}]{} edge from parent node [above]{$a_1$}}
		edge from parent node [above]{$m_2$}
		}
		edge from parent node [right]{$\theta$}
		}
		child[grow=down]{
		node[solid node,label=above left:{\rotatebox{90}{$t_1$}}] {}
		node[solid node,label=below:{Emisor}] {}
		child[grow=left]{node(2)[solid node,label=above:{$[1-p]$}]{}
		child{node[hollow node,label=left:{}]{} edge from parent node [above]{$a_1$}}
		child{node[hollow node,label=left:{}]{} edge from parent node [below]{$a_2$}}
		edge from parent node [above]{$m_1$}
		}
		child[grow=right]{node(4)[solid node,label=above:{$[1-q]$}]{}
		child{node[hollow node,label=right:{}]{} edge from parent node [below]{$a_2$}}
		child{node[hollow node,label=right:{}]{} edge from parent node [above]{$a_1$}}
		edge from parent node [above]{$m_2$}
		}
		edge from parent node [right]{$1-\theta$}
		};
		% information set
		\draw[dashed,rounded corners=10]($(1) + (-.45,.45)$)rectangle($(2) +(.45,-.45)$);
		\draw[dashed,rounded corners=10]($(3) + (-.45,.45)$)rectangle($(4) +(.45,-.45)$);
		% specify mover at 2nd information set
		\node[rotate=90] at ($(1)!.5!(2)$) {$\underset{h_1}{\text{Receptor}}$};
		\node[rotate=90] at ($(3)!.5!(4)$) {$\underset{h_2}{\text{Receptor}}$};
	\end{tikzpicture}
\end{center}

Considera los siguientes datos del juego:

\begin{enumerate}
  \setlength{\itemsep}{0pt}
  \setlength{\parskip}{0pt}
  \setlength{\parsep}{0pt}
  \item Jugadores: Trabajador y Empleador.
  \item Tipos del trabajador: Alto desempeño (Alto), o Bajo desempeño (Bajo).
  \item La probabilidad del tipo Alto es de 0.2, y la del bajo de 0.8.
  \item El conjunto de mensajes del trabajador es $ m = \{\text{Estudiar (E), No Estudiar (NE)}\} $.
  \item Las acciones del Empleador son $ A = \{\text{Adm, Tec}\} $, en donde Adm es de administración, Tec de técnico.
  \item El Empleador ofrece salarios $ w_{\text{Alto}} = 10, w_{\text{Bajo}}=6 $, y tiene ganancias según
  \begin{center}
    \begin{game}{2}{2}[tipo de Trabajador][empleo]
            & Adm 	& 	Tec \\ 
      Alto 	&	12 		& 	7	 \\
      Bajo 	&	1			& 	4
    \end{game}
  \end{center}
  \item El trabajador tiene un costo privado $ c_{\text{Bajo}} = 4 $ y un costo $ c_{\text{Alto}} = 1 $ por estudiar, y \textbf{si no estudia} su costo es 0 para ambos tipos $ c_{\text{Bajo}} = c_{\text{Alto}} = 0$.
  \item Las ganancias del trabajador están dadas por el salario menos el costo, es decir $ w_t - c_t $.
\end{enumerate}
 
\textbf{Resuelve lo siguiente}

\begin{enumerate}
\setlength{\itemsep}{0pt}
\setlength{\parskip}{0pt}
\setlength{\parsep}{0pt}
  \item A partir de los anteriores datos, representa nuevamente el juego actualizando correctamente $ m_1, m_2, t_1, t_2, \theta, 1-\theta, a_1, \text{ y } a_2 $ \textbf{(3 pt)}.
  \item Describe los posibles equilibrios separadores y agrupadores \textbf{(2 pt)}.
  \item Evalúa cada uno de los posibles equilibrios, y concluye cuál(es) es(son) EBP, escribiendo correctamente el perfil de estrategias para ambos, que incluye las creencias del jugador no informado \textbf{(5 pt)}.
\end{enumerate}
  
\end{Exercise}

\begin{center}
  ¡Suerte! 
  
  \LARGE\Cat[1.2] 
\end{center}



\end{document}