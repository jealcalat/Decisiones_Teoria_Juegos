\documentclass[12pt]{article} % se puede cambiar por article
\RequirePackage{amsmath,amsfonts,amssymb,amsthm}
\usepackage{eulervm} % tipografía con soporte para matemáticas
\usepackage{caption}
 
\usepackage{exercise}
\usepackage{graphicx}
\DeclareMathOperator*{\argmax}{argmax}

\newcommand{\vect}[1]{\boldsymbol{#1}}

\usepackage[utf8]{inputenc}
\usepackage[spanish,mexico]{babel}
% sgamex.sty from Rubinstein
\usepackage{sgamex}
\usepackage{forest}
\forestset{qtree/.style={for tree={parent anchor=south, 
           child anchor=north,align=center,inner sep=0pt}}}
          
\usepackage{setspace}
\onehalfspacing % doble espacio: \doublespacing

\title{Solución del Examen de la Unidad 1}
\author{Emmanuel Alcalá}
\date{\today}

\begin{document}
\maketitle

\begin{Exercise}[name={Respuesta}]

  Representación del árbol

  \begin{forest} for tree={s sep=30pt}
    [Regina,circle,edge={->},
      [Nat,edge label={node[midway,left]{S}},edge={->},
        [$640000+q-r$,l=15mm, edge label={node[midway,left]{I, $p=0.1$}}]
        [$1000000 - r$,l=15mm, edge label={node[midway,right]{NI, $0.9$}}]
      ]
      [Nat,edge label={node[midway,right]{NS}},edge={->},
        [640000,l=15mm, edge label={node[midway,left]{I}}]
        [1000000,l=15mm, edge label={node[midway,right]{NI}}]
      ]
    ]
  \end{forest}


 \begin{enumerate}
   \setlength{\itemsep}{0pt}
   \setlength{\parskip}{0pt}
   \setlength{\parsep}{0pt}
   \item $ UE_{\text{Regina}}(NS) = 0.1\sqrt{640000} +  0.9\sqrt{1000000}=980$; el EC es la cantidad $x$ con la que recibiría la misma utilidad que la utilidad esperada, por lo que es la función inversa de la utilidad, $ EC = 980^2 = 960400$.
   \item Regina está \textit{al menos tan bien} (lo que implica que puede estar mejor) comparando el seguro que no comprándolo si 
   \[UE_{\text{Regina}}(S) \geq UE_{\text{Regina}}(NS)\]
   Sabemos que $ UE_{\text{Regina}}(NS) = 980 $, y una vez asegurándose, $ UE_{\text{Regina}}(S) = \sqrt{1000000 - r} $. Despejamos $ r $ elevando al cuadrado ambos lados de la desigualdad y reacomodando:
   \[
   r \leq 100000 - 980^2 = 39600,\ r \in [0, 39600]
   \]
   El máximo que está dispuesta a pagar es 39600.
 \end{enumerate}

\end{Exercise}

\begin{Exercise}[name={Respuesta}]

1. T domina a B para cualquiera estrategias del Jugador 2 $ u_{j1}(T, s_2) > u_{j1}(B, s_2) $.
También, R domina a C para todas las estrategias del Jugador 1 $ u_{j2}(s_1, R) > u_{j2}(s_1, C) $.
Sobreviven las siguientes:

\begin{center}
  \begin{game}{2}{2}[Jugador 1][Jugador 2]
          &   L   &     R \\
      T   &  2,0  &   4,2\\
      M   &  3,4  &   2,3
  \end{game}
\end{center}

2. Aplicando el algoritmo de tres pasos, queda
\begin{center}
  \begin{game}{2}{2}[Jugador 1][Jugador 2]
          &   L   &     R \\
      T   &  2,0  &   \underline{\textbf{4,2}}\\
      M   &  \underline{\textbf{3,4}}  &   2,3
  \end{game}

\end{center}
  
El EN es $ \{\underbrace{(M, L)}_{\text{J1}}, \underbrace{(T, R)}_{\text{J2}}\} $. Si 2 juega L, la MR de 1 es M; si 2 juega R, la MR de 1 es T.
  
\end{Exercise}

\begin{Exercise}[name={Respuesta}]

  \begin{center}
    \begin{game}{3}{3}[Trabajador 1][Trabajador 2]
                    &               Aplicar E1                  & Aplicar E2 & $\sigma_i$ \\
        Aplicar E1  & $\frac{1}{2}w_1, \frac{1}{2}w_1$ & $w_1, w_2$ & $p$  \\ 
        Aplicar E2  & $w_2, w_1$                       & $\frac{1}{2}w_2, \frac{1}{2}w_2$ & $1-p$ \\
        $\sigma_j$  & $q$   &   $1-q$ & 
    \end{game}
\end{center}

\begin{enumerate}
  \setlength{\itemsep}{0pt}
  \setlength{\parskip}{0pt}
  \setlength{\parsep}{0pt}
  \item T1 juega con un $ p $ tal que T2 sea indiferente entre E1 y E2, lo que se logra si $ UE_{T2}(E1) = UE_{T2}(E2) $. Esto es:
  \begin{align*}
    UE_{T2}(E1) &= UE_{T2}(E2)\\
    \text{a la izquierda para E1, }& \text{a la derecha para E2}\\ 
    p(1/2)w_1 + (1-p)w_1 &= pw_2 + (1-p)(1/2)w_2\\
    pw_1/2 + pw_2/2 &= w_1 - w_2/2\\
    \text{multiplicamos todo por 2, } & \text{factorizamos y despejamos para } p\\
    p &= \frac{2w_1 - w_2}{w_1 + w_2}
  \end{align*} despejando
  El EN en estrategias mixtas  (ENm) es, para ambos jugadores,  $ \left\{\left ( \frac{2w_1 - w_2}{w_1 + w_2}, 1 -  \frac{2w_1 - w_2}{w_1 + w_2}\right ), \left ( \frac{2w_1 - w_2}{w_1 + w_2}, 1- \frac{2w_1 - w_2}{w_1 + w_2}\right )\right\} $.
  \item Si tenemos
  \begin{center}
    \begin{game}{2}{2}[Trabajador 1][Trabajador 2]
                    &               Aplicar E1                  & Aplicar E2 \\
        Aplicar E1  & 5, 5 &  10, 8 \\ 
        Aplicar E2  & 8, 10 & 4, 4
    \end{game}
\end{center}
  La utilidad del jugador 1 es simplemente 
  \[
    u_1(\sigma_1, \sigma_2) = pq(5) + p(1-q)(10) + (1-p)q(8) + (1-p)(1-q)4 = 60/9
  \]
  Una forma más fácil de resolverlo es como le producto punto vector-matriz-vector (expuesto en las notas). Con $ A $ con una matriz de los pagos de T1 (ordenados según las filas):
  \[
    u_1(\sigma_1, \sigma_2)= \vect{\sigma}_1 A\vect{\sigma}^T = %
    \begin{bmatrix} 
      2/3 & 1/3
    \end{bmatrix}\times
    \begin{bmatrix} 
      5 & 10 \\
      8 & 4
    \end{bmatrix}\times 
    \begin{bmatrix} 
      2/3 \\ 1/3
    \end{bmatrix} = 6.666 \approx 60/9
  \]
  
\end{enumerate}
  
\end{Exercise}

\end{document}